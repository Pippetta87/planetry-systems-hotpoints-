\documentclass[main.tex]{subfiles}

\begin{document}

\newrefcontext[sorting=hot]
%\printbibliography[filter={terrestrialplanets},keyword={primary},heading=bibintoc,title={\textcolor{beaver}{Biblio about: ''Sistema solare: pianeti terrestri''}}]
%\printbibliography[filter={terrestrialplanets},notkeyword={primary},heading=bibintoc,title={\textcolor{beaver}{Other refs about: ''Sistema solare: pianeti terrestri''}}]

{\let\clearpage\relax
\chapter{Modelli di (e vincoli) formazione del sistema solare}
}

\begin{refsection}[solarsystem.bib]
\begingroup
\nocite{*}
\let\clearpage\relax
%\settoggle{bbx:note}{false}
%\AtNextBibliography{\AtEveryBibitem{\clearlist{note}}}
\boolfalse{isreasearch}
\printbibliography[filter={pps},heading=bibintoc,title={\textcolor{antiquefuchsia}{Vincoli a formazione planetaria da modelli/osservazioni formazione sistema solare}}]
\booltrue{isreasearch}
\printbibliography[filter={pps},check=bibsearchdef,heading=bibintoc,title={\textcolor{antiquefuchsia}{Bib research}}]
\endgroup
\end{refsection}

\section{CAI, Chondrules etc}

{\let\clearpage\relax
\chapter{Struttura orbitale sistemi planetarii}
}

\begin{refsection}[orbitalevolution.bib,migration.bib]
\begingroup
\nocite{*}
\let\clearpage\relax
\printbibliography[filter={pps},heading=bibintoc,title={\textcolor{antiquefuchsia}{Evoluzione orbitale in PPD e dopo che \'e dissipato}}]
\endgroup
\end{refsection}

\section{Migration and (lack of) mean motion resonance capture}
\cite{batygin2015capture}: integrable hamiltonian formalism to predict success/failure of resonant capture
\subsection{Context: resonances in solar system and exo-systems}

\begin{itemize}
\item modest predisposition for orbital resonance: exoplanet radial-velocity shows overabundancy of resonant giant planet, sub-jovian kepler planet distro shows enhacement outside $2:1$, $3:2$ commensurabilities
\item processes that can slowly bring orbits closer: tidal dissipation (\cite{goldreich1966q}, \cite{yoder1981tides}, \cite{peale1976orbital} (and analytic development))    
\item Hamiltoniana risonante: perturbation to keplerian motion, planet-planet potential is expanded in Fourier series in orbital angles and power series in inclinations and eccentricities
\end{itemize}

\subsection{3bodies problem: analytical theory}

\subsection{Application to solar system}
\begin{itemize}
    \item 
\end{itemize}
\subsection{coplanar three body}

\subsection{Star-planet interaction: tide, }

\begin{itemize}
    \item \cite{benbakoura2019evolution}: SL star-planet secular evolution of stellar of stellar rotation and orbital params under magnetic braking and tidal dissipation - ESPEM code (). Star-planet interaction involve stellar structure, rotation, magnetism, planet orbital parameters; spin down increases corotation radius until becomes larger than $a_p$ - planets spiraling down can spin up star (magnetism: Brun 04 15) - compute evolution of star-planet systems (Ferraz-Mello 15) 
\end{itemize}

\end{document}